\documentclass{article}
\usepackage{amsmath,amsfonts,amssymb,url}
\begin{document}

\textbf{Call For Partners: Romance with Rigor}\\
\textit{Industry Track}
\author{Brandon Bohrer}
%acknowledge ram, stefan, ryan, brandon
\abstract{}

\section{Introduction}
Traditional dating sites are based on simple premise: participants are capable of identifying, with a reasonable degree of accuracy, which of their proposed matches are appropriate for them.
Millions of years of empirical dating evidence show this premise to be false.
As in all human pursuits, the dating field is rife with biases and irrational judgements, such as ``You're too ugly'', ``you don't make enough money'', and ``at least he can remember my birthday, Brandon.''
While no system can completely eliminate human biases, the scientific community has, by and large, done an impeccable job of minimizing the influence of bias through its system of \emph{peer review}. Even in non-blind review, simply entrusting the review process to a disinterested third party with no conflicts of interest greatly increases the quality of the outcome. If only the same could be said for love.

Unlike all other dating sites, \url{callfor.partners} addresses the \emph{underlying} problem in online dating: lack of rigor.
We do so by applying that most successful of human inventions, peer review.
As with academic peer review, not all decisions are made by peers.
Just like you get to choose what paper to submit, \textbf{YOU} are in charge of who you want to date, by writing your own personal \textbf{CALL FOR PARTNERS} and \textbf{YOU} get to make the best impression by submitting your own \textbf{PARTNERSHIP ABSTRACTS}.
Only the messiest, most error-prone part is spread between peers: Deciding which partnerships to pursue and which to reject.
The part that nobody wanted to do anyway.

\section{Design}
The central feature of the Call For Partners Partnering Workflow is the \emph{Partnering Committee}.
As with the peer review process, one selects a group of one's closest friends to make decisions as to which advances should be accepted or rejected.
Usually (but not always) the reviewing process is mutual: users are motivated to join their friends' partnering committees for the reviewing services they themselves receive in return. The standard dating website trope of ``profiles'' appears in Call For Partners under the guise of Calls For Partners.
The distinguishing feature of a CFP vs. the traditional profile is that unlike a traditional profile, a CFP is all about what \textbf{YOU} want, specifying in utmost precision the desired features in a partner. CFPs come in multiple styles. For example, a \emph{Journal CFP} often has a rolling deadline or no deadline at all, where potential partners are encouraged to submit at whatever time they find convenient. In contrast, a \emph{Conference CFP} generally has a strict deadline which is extended only after a disappointingly small number of submissions. The Conference CFP is especially useful for implementing \emph{rebound relationships}, a implemented on many sites, but never before with such rigor.

The CFP model acknowledges that you are TALENTED, and your history will speak for itself. In addition to a CFP, every participant has a CV listing relevant accomplishments. Because we are living in the future, Call For Partners applies advanced AI technology known as ``Facebook Stalking'' to automatically generate large portions of a CV. This open-access model (with a level of openness exceeding many leading-edge academic organizations including SIGPLAN) reduces harrassment and other abuses of the system, because participants are held accountable for their behavior in public. Remember, boys: \textbf{Before you do something you'll regret, DBLP don't forget}.

Given a CFP and CV, particpants have all the information necessary to write a Partnership Abstract.
A partnership abstract gives a brief, concise description of the contents of the proposed relationship.
The Partnership Committee compares the abstracts against each author's CV and own CFP, and uses this to write reviews, ranking each abstract and/or each individual's self-worth on a scale of A-F.
In most cases, the reviews are clear enough that the PC can reach an anonymous conclusion as to whether the abstract should be accepted or not.
In the case of a dispute, a PC Chair can be appointed with tie-breaking authority.

Upon acceptance, the relevant parties gain the ability to message each other
Often, the author of the CFP will request revisions from the author of the Partnership Abstract before starting the Parternship.
While the proposer does not have the ability to reject the proposee nor request changes of them, it is traditional to disclose further results publicly, which over time has a significant affect on the proposee's Impact Factor.

As with most dating sites, CFP provides funcitonality to help you search through potential matches into to identify someone to whom you wish to submit a Partnership Abstract. In addition to the barebones necessities like salary, race, and number of previous partners, CFP allows you to perform custom searches that solve the problems specific to academic communities. In fact, we provide the first known solution to the Two-Body Problem, a generalization of the Three-Body Problem initially posed by Newton.

The Two-Body Problem was originally phrased by Newton is stated as follows:
\textbf{Given two bodies A and B each with an attraction and a PhD, find a place of residency and a salary.}

The key issue here, of course, is to identify a salary for two PhD's at the same time, as only so many employers are willing to employ the unemployable.
In offline dating, it has been widely recognized that the problem becomes significantly easier as the distance in thesis topics increases.
Thus, the Two-Body problem can be reduced to the Some-Body Problem as posed by Mercury:
\textbf{Can anybody find me somebody to love? Who has a PhD in a technical field, but preferably not CS and definitely not formal methods, oh god please not formal methods?}

At this point, the astute reader will notice that this problem is amenable to solution via an adequate search feature.
The first-of-its-kind CFP Field Search allows one to specify the exact desired distance between their mate's work and their own, using traditional metrics such as Er\''os Numbers, Bacon Numbers, Erdos-Bacon Numbers, and also less traditional metrics. For example some Jewish members of the community have found the Erdos-Matzah number useful for kosher observances.

Because we provide the first and only solution for such a seminal problem in bodyology, we provide a formal model and proof of our solution to the two-body problem in Differential Dynamic Logic, an established logic for verifying hybrid systems, many of which are Cyber-Physical Systems.
This marks its first known use in verifying Cyber-Social-Physical-;-)-Academic Systems.

\subsection{Model of Academics in Motion}
\subsection{Proof}
% Lemma: Bounded attraction
% Lemma: Stable orbit
% Lemma: Stable distance
% Lemma: Density of Jobs
% Theorem: Salary



\section{Implementation}
Call For Partners is currently available to the public as an open beta at \url{callfor.partners}.
Call For Partners is implemented with Scala, Slick, Play, PostgreSQL, Heroku and assorted other buzzwords.



\section{Evaluation}

\section{Testimonials}

\section{Related Work}
Many websites have addressed the related work of dating, though the author strongly suggest that you do not date your relatives for it is bad for the gene pool.
Websites such as eHarmony are based on the pseudo-science of compatibility.
Anyone who has ever tried to open a Word 2016 in Word 95 knows that most things we say are compatible, are not, and thus it is with humans.
OkCupid appeals to paganistic rituals in the hopes of receiving optimal pairings from the divines, who, unfortunately, do not believe in computers.
Match.com and Tinder both determine optimal dates by lighting large groups of singles on fire and seeing which ones burn to the ground.
Those which do not burn are witches and thus not good dating material, but unfortunately by that point all the good ones are dead, making the method ineffective in practice.

\section{Promotional Offers}

\section{Conclusion}

\end{document}